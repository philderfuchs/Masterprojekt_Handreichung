
\section{Color Palette Estimation}
\label{sec:cpe}

In diesem Abschnitt wird eine Algorithmus zur Lösung des Teilproblems $f_{CPE}$ ermittelt. Dies bedeutet die Festlegung eines Verfahrens zur Abbildung $I \to P$, wobei $P = (c_1, ..., c_n)$ die Bildvorlage $I$ mit einer minimalen Menge von Farben repräsentiert. \autoref{sec:cpe-ueberblick} bietet einen Überblick über vorhandene Algorithmen zur CPE. Im Anschluss wird  eine konkrete Methoden ausgewählt und in \autoref{sec:cpe_acopa} erläutert.

\subsection{Überblick}
\label{sec:cpe-ueberblick}

Grundlegend sind zwei Ansätze zur CPE zu unterscheiden:
\begin{enumerate}
    \item \textbf{Histogramm-basiert}: Algorithmen, die von der räumliche Anordnung der Farben im Bild abstrahieren und $P$ auf Grundlage der globalen Farbinformation bilden, d.h. des Histogramms. Da der Definitionsbereich eines Histogramm eine Teilmenge des Farbraums bildet, entspricht dieser Fall der unüberwachten Klassifizierung des Farbraums. Die Umsetzung erfolgt überwiegend durch Clustering-Verfahren, die Gruppen ähnlicher Farben im Bild identifizieren.
    \item \textbf{Segmentierungs-basiert}: Algorithmen, die die räumliche Anordnung der Farben im Bild berücksichtigen. Durch eine Segmentierung des Bildes werden zusammenhängende Komponenten identifiziert, für welche im Folgenden repräsentative Farben ermittelt werden. Die menschlichen Wahrnehmungseigenschaften auf Komponentenebene werden auf diese Weise besser berücksichtigt, gleichzeitig wird durch die zusätzliche Betrachtung der Positionsinformation eine weitere Ebene der Komplexität eingeführt \citep{colorthemes}.
\end{enumerate}

\citet{image-based-schemes} kritisieren an den segmentierungs-basierten Verfahren, dass sie von einer akkuraten Bildsegmentierung abhängen, welche nicht für jede Bildstruktur gewährleistet ist (z.B. in stark texturierten Bereichen). Aus diesem Grund wird für die weitere Diskussion der Fokus auf die histogramm-basierten Verfahren gelegt.

\citet{categorization} treffen eine Kategorisierung der histogramm-basierten Verfahren in \emph{hierarchisch} und \emph{iterativ}. Hierarchische Verfahren starten vor dem Erreichen der (fest zu wählenden) Farbanzahl $n$ mit mehr (\emph{bottom-up}) bzw. weniger (\emph{top-down}) Clustern. Die Aufspaltung bzw. Vereinigung von Clustern basiert auf der statistischen Analyse der Verteilung der Bildfarben im Farbraum. Hierzu zählen unter anderem die in der Vergangenheit populären top-down Raumunterteilungs-Algorithmen wie z.B. Mediancut \citep{mediancut} oder Octree \citep{octree}, welche den Farbraum sukzessiv in disjunkte Teilräume zerlegen und dabei den Clustern eine Würfelform unterstellen. Hierin begründet ebenfalls die Kritik an diesen Verfahren. Ergebnis der Verarbeitung ist ein Dendogram, dessen Blätter die Farben der Farbpalette repräsentieren.

Ein Schnitt des Dendograms entspricht einer Partitionierung des Raums, was dem Ergebnis der iterative Verfahren entspricht.  Sie starten bereits mit der erforderlichen Anzahl Clustern $n$ und optimieren deren Position iterativ entsprechend einer Zielfunktion. Ein Beispiel ist die Minimierung des quadratischen Fehler, wie z.B. bei K-Means \citep{kmeans, kmeanshsi} oder Fuzzy C-Means \citep{fuccycmeans}. Andere Verfahren analysieren das Histogramm auf dichte bzw. weniger dichte Regionen, wie z.B. der Mean-Shift Algorithmus \citep{meanshift}. Die Kritik der iterativen Verfahren begründet sich in deren Abhängigkeit von den Startbedingungen \citep{acopa}.

Für eine detailliertere Vorstellung von Algorithmen zur CPE verweist der Autor auf \citet{categorization2}.

\begin{figure}[h]
\centering
\includegraphics[width=0.48\textwidth]{img/peppers.png}
\caption{CPE Ergebnis von ACoPa. (a) Originalbild "Peppers". (b) Hierarchische Farbpalette. Die oberste Ebene identifiziert die wesentlichen Farbtöne. Ebene 2 bildet Sättigungsstufen. Ebene 3 schlüsselt das Ergebnis weiterhin in Helligkeitsstufen auf und zeigt die finalen Farben. (Quelle: \citep{acopa})}
\label{fig:peppers}
\end{figure}

\citet{acopa} kritisieren an den bisherigen Algorithmen, dass die Anzahl $n$ gesuchter Farben zuvor bekannt sein muss und dass Farben kleiner Bilddetails im Sinne der Definition aus \ref{sec:bildvorlagen} nur unzureichend repräsentiert werden. Dies begründet sich in der Tatsache, dass die bisherigen Verfahren eine Raumunterteilung vornehmen, ohne die Spezifika von Farbräumen zu beachten.

Demgegenüber stellen \citet{acopa} und \citet{image-based-schemes} zwei ähnliche, hierarchische Verfahren vor, die sich die Verwandtschaft der menschlichen Farbbeschreibung mit der Farbkodierung in Farbräumen mit Polarkoordinaten-Repräsentation zu Nutze machen. Sie basieren auf der hierarchischen Segmentierung eindimensionaler Histogramme. Die Algorithmen ermitteln zunächst die grundlegenden Farbtöne (Hue) der Bildvorlage, welche daraufhin sukzessive in Sättigungsstufen (Saturation) und diese wiederum in Helligkeitsstufen aufgespalten werden. Abbildung \ref{fig:peppers} veranschaulicht exemplarisch die hierarchische Arbeitsweise. Die Identifizierung verschiedener Farbtöne der Bildvorlage vereinfacht die Bildung \emph{monochromer}, \emph{dualer} und \emph{triadischer} Farbschemen, welche in \autoref{sec:farbschemata} vorgestellt wurden. Darum werden diese beiden Algorithmen für die weitere Diskussion ausgewählt.

Der von \citet{acopa} vorgeschlagene Algorithmus lautet  \textbf{Automatic Color Palette (ACoPa)} und verwendet den HSI-Farbraum. Er arbeitet parameterfrei und bestimmt $n$ automatisch. Der Algorithmus von \citet{image-based-schemes} verwendet den HLS-Farbraum und ist parameterabhängig, d.h. er benötigt Angaben in Bezug auf $n$. Die Autoren schlagen jedoch interessante Vorverarbeitungsschritte für die Histogramm-Bildung vor, welche die menschlichen Wahrnehmungseigenschaften stärker berücksichtigen und somit eine Brücke zu den segmentierungs-basierten Verfahren schlagen.

Um dem Ziel einer vollautomatisierten Methode zur Farbgestaltung von Webseiten aus Bildvorlagen gerecht zu werden, wird als Implementierungsgrundlage der ACoPa-Algorithmus gewählt, da er keine Nutzerinteraktion in Form der Bestimmung von $n$ benötigt. Er wird im folgenden \autoref{sec:cpe_acopa} vorgestellt. Daraufhin werden in \autoref{sec:vorverarbeitung} die Adaptionen dieses Algorithmus von \citet{image-based-schemes} beschrieben.



\subsection{ACoPa}
\label{sec:cpe_acopa}

Der ACoPa Algorithmus \citep{acopa} basiert auf der Darstellung des Histogramms im HSI Raum. Die Konvertierung wird in Abschnitt \ref{sec:hsi-raum} vorgestellt. Daraufhin werden Farben identifiziert, indem Segmentierungen der eindimendionalen H-, S- und I-Histogramme durchgeführt werden. Das zugehörige Verfahren wird als \emph{Fine-to-Coarse Segmentation} bezeichnet und in Abschnitt \ref{sec:segmentierung} vorgestellt. Abschnitt \ref{sec:hierarchische-farbpalette} bespricht die hierarchische Anwendung dieses Verfahrens auf die Komponenten des Farbraums zur Bildung der Farbpalette und stellt Ergebnisse vor.

\subsubsection{Konvertierung in den HSI-Raum}
\label{sec:hsi-raum}

\begin{figure}
\centering
\includegraphics[width=0.5\textwidth]{img/hsi_conversion.png}
\caption{Visualisierung des HSI Farbraums.}
\label{fig:hsi_conversion}
\end{figure}


Zunächst wird das Histogramm in den HSI Farbraum mit $\{(h, s, i) \ | \ 0 \leq h < 360 \wedge 0 \leq s, i \leq 1\}$ übertragen. Es handelt sich um eine Rotation des RGB-Würfels, der die Kodierung von Farbwerten in Polarkoordinaten ermöglicht. Die einzelnen Komponenten lauten Hue (Farbton), Saturation (Sättigung) und Intensity (Intensität, hier als Helligkeit bezeichnet). Die Umrechnungsvorschrift gemäß der ACoPa-Autoren lautet wie folgt:

\begin{equation}
\begin{split}
I = \frac{R+G+B}{3} \\
S = \sqrt{(R-I)^2 + (G-I)^2 + (B-I)^2} \\  
H = \arccos{(\frac{(G-I)-(B-I)}{S\sqrt{2}})}
\end{split}
\label{eq:hsi_acopa}
\end{equation}


Das Umrechnungsergebnis führt zu einer falschen Abbildung der H-Werte. \citet{colorimage} schlagen eine alternative Bildungsvorschrift der H-Komponente vor:

\begin{equation}
\begin{split}
H = \arccos{(\frac{\frac{1}{2}((R-G)+(R-B))}{\sqrt{(R-G)^2+(R-B)(G-B))}})}
\end{split}
\label{eq:hsi_colorimage}
\end{equation}


\autoref{fig:hsi_conversion} visualisiert des Konvertierungsergebnis. Das Resultat ist ein RGB-Würfel, der auf die \glqq{}schwarze Ecke gestellt wurde\grqq{}.
 
\subsubsection{Segmentierung des Histogramm}
\label{sec:segmentierung}

Ein Histogramm $h=(h_i)_{i = 1 \ldots b}$ mit b-Bins wird gebildet. Gesucht wird nun eine Sequenz $s = (s_i)_{i = 1 \ldots k}$ von Markierungen des Definitionsbereichs mit $1 = s_0 < s_1 < \ldots < s_k = b$, welche als \textbf{Segmentierung} des Histogramms bezeichnet wird. Das Intervall $[h_{s_i}, \ldots,  h_{s_{i+1}}]$ wird als \textbf{Segment} bezeichnet. Die Vereinigung aller Segmente eines Histogramms ergibt wieder das Histogramm. Ziel ist, dass das Histogramm in allen Segmenten eine \glqq{}annähernd unimodale Verteilung aufweist\grqq{} \citep{acopa}. Abbildung \ref{fig:unimodal} zeigt das Prinzip an verschiedenen Beispielen.

\begin{figure}[h]
\centering
\includegraphics[width=0.48\textwidth]{img/unimodal.png}
\caption{Beispiele der Segmentierung von Histogrammen in unimodale Abschnitte. (Quelle: \citep{acopa})}
\label{fig:unimodal}
\end{figure}

Ein Histogramm ist offensichtlich in jedem Segment unimodal, wenn $s$ mit den enthaltenen Minima initialisiert wird. Es wird nun versucht, Komponenten aus $s$ zu entfernen, indem für ein bestimmtes $i \in s_2, ..., s_{k-1}$ überprüft wird, ob $h$ im Intervall $[h_{s_{i-1}}, \ldots,  h_{s_{i+1}}]$ die \glqq{}unimodale Hypothese\grqq{} erfüllt. Anschaulich bedeutet dies die Verschmelzung benachbarter Segmente, so dass das neu entstandene Segment nach wie vor \glqq{}annähernd unimodal ist\grqq{}. Hierfür wird $h$ im betrachteten Intervall mit einem Referenz-Histogramm $h'$ verglichen. Es wird ein $j \in \; ]s_{i-1}, s_{i+1}[$ gewählt, so dass $h'$ in $[h'_{s_{i-1}}, \ldots,  h'_j]$ monoton wachsend und in $[h'_j, \ldots,  h'_{s_{i+1}}]$ monoton fallend ist, d.h. in $[h'_{s_{i-1}}, \ldots,  h'_{s_{i+1}}]$ unimodal. Das Referenz-Histogramm $h'$ wird aus dem Original-Histogramm $h$ durch Anwendung des \emph{Pool Adjacent Violators} Algorithmus im betrachteten Segment gebildet. \autoref{fig:grenander} zeigt ein Beispiel für die monoton fallende Variante. Die Bildungsvorschrift für diese Transformation ist \citep{acopa, ftc} zu entnehmen. 

Für $\forall j \in \; ]s_{i-1}, s_{i+1}[$ werden $h_{[s_{i-1}, s_{i+1}]}$ und $h'_{[s_{i-1}, s_{i+1}]}$ durch einen parameterfreien statistischen Test verglichen, den die Autoren in einer separaten Veröffentlichung vorstellen \citep{ftc}. Ziel ist der Nachweis, dass $h$ im betrachteten Segment die \glqq{}unimodale Hypothese\grqq{} erfüllt, d.h. $h_{[s_{i-1}, s_{i+1}]}$ und $h'_{[s_{i-1}, s_{i+1}]}$ eine ähnliche Verteilung aufweisen. Bei Erfolg wird $s_i$ aus $s$ entfernt. Der von den Autoren vorgestellte Test ist verhältnismäßig aufwändig. In dieser Arbeit wird für die Implementierung auf einen einfachen T-Test zurückgegriffen. Dieser liefert ebenfalls befriedigende Ergebnisse, ist aber abhängig vom gewählten Signifikanzniveau, welcher experimentell ermittelt wurde.

\begin{figure}[]
\centering
\includegraphics[width=0.48\textwidth]{img/grenander.png}
\caption{Transformation eines Segments durch Anwendung des \emph{Pool Adjacent Violators} Algorithmus. Links: Betrachtetes Segment. Rechts: Monoton fallende Variante. (Quelle: \citep{acopa})}
\label{fig:grenander}
\end{figure}

Das Verfahren zur Histogramm-Segmentierung wird in \citep{ftc} als \textbf{Fine-to-Coarse (FTC) Segmentation} Algorithmus zusammengefasst. Zunächst wird $s$ mit allen Minima des Histogramms initialisiert. Daraufhin werden so lange benachbarte Segmente durch Überprüfung der unimodalen Hypothese verschmolzen, bis keine Verschmelzung mehr möglich ist. Eine repräsentierende Farbe eines Segments wird durch Mittelung der enthaltenen Samples gebildet. Abbildung \ref{fig:h_segmentation} zeigt dies an einem Beispiel.

\begin{figure}[]
\centering
\includegraphics[width=0.95\textwidth]{img/h_segmentation.png}
\caption{Beispiel für eine Segmentierung des Hue-Histogramms. (a) Bildvorlage, Albumcover \glqq{}My Beautiful Dark Twisted Fantasy\grqq{} von Kanye West. (b) Hue-Histogram (normalisiert), mit allen Minima als initiale Segmentierung. (c) Segmentierung nach Anwendung des FTC Segmentation Algorithmus. (d) Farbmittelpunkte entsprechend der Samples der jeweiligen Segmente.}
\label{fig:h_segmentation}
\end{figure}

\subsubsection{Bildung der hierarchischen Farbpalette}
\label{sec:hierarchische-farbpalette}

Der ACoPa Algorithmus besteht aus einer hierarchischen Anwendung des FTC-Segmentation Algorithmus. Dabei wird zuerst der $H$-, danach der $S$- und abschließend der $I$-Kanal segmentiert. In jedem Schritt werden die Samples der entstandenen Segmente aufgeteilt und die Histogramme der nächsten Ebene separat berechnet. Das Ergebnis ist eine hierarchische Farbpalette. Abbildung \ref{fig:palette} zeigt dies am Beispiel der Covers aus Abbildung \ref{fig:h_segmentation}. Auf oberster Ebene (H) werden die grundsätzlichen Farbtöne des Bildes identifiziert. Auf der zweiten Ebene (S) werden die jeweiligen Farbtöne in unterschiedliche Sättigungsstufen aufgeteilt. Auf der dritten Ebene (I) werden von den Sättigungen verschiedene Helligkeitsabstufungen gebildet.

Die letzte Ebene (I) bildet die Farbpalette $P$. Sie wird durch die initial erschlossenen Farbtöne in \textbf{Hue-Groups (HG)} unterteilt. Somit wird für $\forall c \in P$ gleichzeitig erfasst, welche Farben Varianten eines gemeinsamen Farbtons sind. \citet{acopa} empfehlen zusätzlich, die Farben als Startpunkte für den K-Means Algorithmus zu verwenden. Abbildung \ref{fig:palette} (b) zeigt das Ergebnis nach Konvergenz. Obwohl die sich die Cluster den im Bild enthaltenen Farben angenähert haben, ist zu sehen, wie hierdurch die Zuordnung zu den Hue-Groups verloren geht. Entsprechend der Argumentation in \autoref{sec:cpe-ueberblick} ist diese Zuordnung jedoch günstig zur Bildung von Farbschemen. Aus diesem Grund wird auf die nachgeschaltete Anwendung des K-Means Algorithmus verzichtet.

\begin{figure}[h]
\centering
\includegraphics[width=0.95\textwidth]{img/palette.png}
\caption{(a) Hierarchische Farbpalette des Covers aus Abbildung \ref{fig:h_segmentation}. Die 14 Farben sind in 4 Hue-Groups angeordnet. Die I-Ebene bildet die Farbpalette der Bildvoralge. (b) Farbpalette nach Anwendung von K-Means.}
\label{fig:palette}
\end{figure}

\subsection{Vorverarbeitung des Histogramms}
\label{sec:vorverarbeitung}


