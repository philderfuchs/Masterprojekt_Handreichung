\section{Fazit und Diskussion}

In dieser Arbeit wurde ein Verfahren für die automatisierte Farbgestaltung von Webseiten aus Bildvorlagen vorgestellt. Die Oberflächenelemente sollen so eingefärbt werden, dass sie die Farben einer Bildvorlage wiederspiegeln. Gleichzeitig wird eine sinnvolle Zuordnung der Farben angestrebt, die Lesbarkeit und Benutzerführung berücksichtigt. Durch die Analyse von Gestaltungsprinzipien im Web- und Screendesign wurde eine exemplarische Menge von Funktionsgruppen herausgearbeitet, die dem Zweck dienen, die Elemente der Weboberfläche visuell voneinander abzugrenzen, zu priorisieren und einheitliche Funktionen zu signalisieren. Durch Modellierung eines Constraint Graphen wurden Kriterien formuliert, die für Farben der Funktionsgruppen gelten sollen. Die Menge möglicher Farbwerte wird durch eine Farbpalette definiert, welche durch den ACoPa-Algorithmus von \citet{acopa} aus der Bildvorlage ermittelt wird.

Charakteristisch für die vorgestellte Lösungsmethode ist die Aufteilung in zwei getrennte Suchverfahren. Eines dient der Ermittlung der Farbpalette (Color Palette Estimation, CPE) und eines der Zuordnung der Farben zu den Funktionsgruppen (Farbschemabildung). Diese Entscheidung wurde getroffen, um die Wiederverwendbarkeit einer Farbpalette für Webseiten mit anderen Funktionsgruppen zu gewährleisten. Für zukünftige Arbeiten bedeutet dies, dass beide Teilprobleme unabhängig voneinander erweiterbar sind.

Eine mögliche Erweiterung für die Farbschemabildung stellt die Behandlung von Fällen dar, in der keine Farbkombination existiert, die alle Hard Constraints erfüllt. Eine Variante hierfür wurde bei der Ermittlung eines monochromen Farbschemas in  \autoref{sec:monochrome} vorgestellt, wobei die Komplementärfarbe ergänzt wurde. Die intelligente Erweiterung der Farbpalette um Farben, die eine Lösung des Constraint Systems ermöglichen, stellt einen allgemeinen Lösungsansatz für dieses Problem dar. Auf diese Weise können ebenfalls Farben berücksichtigt werden, die Teil des Corporate Designs eines Webauftritts sind.

Eine weitere Möglichkeit zur Erweiterung der vorgestellten Methode ist die Berücksichtigung des räumlichen Bezugs der Oberflächenelemente. In der präsentierten Lösung wurde pauschal ein ausreichender Kontrast von Farbkombinationen erzwungen, bei denen bekannt war, dass sie als Text- und Hintergrundfarbe vorkommen. Eine Auswertung der räumlichen Beziehungen würde das Farbschema besser auf das tatsächliche Layout anpassen.